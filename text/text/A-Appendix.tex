%!TEX root = ../main.tex
\section{Der Service Manager als dienstorientierte Plattform}
\subsection{Sequenzdiagramm zur Migration im Service Transporter}
\begin{figure}[!htb]
	\centering	
	\includegraphics[width=0.7\textwidth]{graphics/sequence_diagram_service_transporter_full.pdf}
	\caption[Sequenzdiagramm zur Migration im Service Transporter]{Sequenzdiagramm zur Migration im Service Transporter}
	\label{fig:sequenzdiagramm_komplett_service_transporter}
\end{figure}
\newpage
\section{Die Evaluation der Dienstemigration im Internet der Dinge}
\subsection{Vollständiger Graph des MIOT-Testbeds zur Emulation eines IoTs}
\begin{figure}[!h]
	\centering	
	\includegraphics[width=\textwidth, trim=1.3cm 1.3cm 1.3cm 1.3cm, clip]{graphics/full_graph_circo.pdf}
	\caption[Vollständiger Graph des MIOT-Testbeds zur Emulation eines IoTs]{Vollständiger Graph des MIOT-Testbeds zur Emulation eines IoTs}
	\label{fig:vollstaendiger_graph_miot_emulation_iot}
\end{figure}
\newpage
\subsection{Beschnittener Graph des MIOT-Testbeds zur Emulation eines segmentierten IoTs}
\begin{figure}[!h]
	\centering	
	\includegraphics[width=\textwidth, trim=1.3cm 1.3cm 1.3cm 1.3cm, clip]{graphics/seperated_graph_circo.pdf}
	\caption[Beschnittener Graph des MIOT-Testbeds zur Emulation eines segmentierten IoTs]{Beschnittener Graph des MIOT-Testbeds zur Emulation eines segmentierten IoTs}
	\label{fig:beschnittener_graph_miot_emulation_segmentiertes_iot}
\end{figure}
\clearpage

\section{Inhalt der CD}
\label{appendix:cd}
\resizebox{0.9\textwidth}{!}{
	\begin{forest}
		pic dir tree,
		where level=0{}{% folder icons by default; override using file for file icons
			directory,
		},
		where n children=0{%
			if level=1{}{%
				before drawing tree={y+=4mm}
			}
		}{%
		}
		[CD
		[Bin]
		[Paper
		[\textcolor{gray}{\textit{beinhaltet alle im Text referenzierten Veröffentlichungen}},
		desc, no edge,
		before drawing tree={x-=4mm,y-=2mm}]
		]
		[Presentation
		[\textcolor{gray}{\textit{beinhaltet die Abschlusspräsentation des Kolloquiums}},
		desc, no edge,
		before drawing tree={x-=4mm,y-=2mm}]
		[presentation.pdf, file]
		[presentation.tex, file]
		]
		[Results
		[\textcolor{gray}{\textit{beinhaltet die Ergebnisse der Messungen aus Kapitel 3 und 5}},
		desc, no edge,
		before drawing tree={x-=4mm,y-=2mm}]
		[final\_test]
		[reachability\_test]
		[throughput\_test]
		]
		[Src
		[1\_servicemananger
		[\textcolor{gray}{\textit{Quellcode der gesamten Service Mananger Plattform}},
		desc, no edge,
		before drawing tree={x-=4mm,y-=2mm}]
		[server]
		[test]
		[logging.conf, file]
		]
		[2\_networktopology
		[\textcolor{gray}{\textit{Quellcode der in Kapitel 3 durchgeführten Messungen und Auswertungen}},
		desc, no edge,
		before drawing tree={x-=4mm,y-=2mm}]
		[reachability\_test.py, file]
		[topology\_evaluation.py, file]
		]
		[3\_performance
		[\textcolor{gray}{\textit{Quellcode des in Kapitel 5 verwendeten Dienstes und Clients}},
		desc, no edge,
		before drawing tree={x-=4mm,y-=2mm}]
		[performance\_client.py, file]
		[performance\_service.py, file]
		]
		[4\_mininet
		[\textcolor{gray}{\textit{Quellcode des in Kapitel 5 erstellten Mininet Netzwerkes}},
		desc, no edge,
		before drawing tree={x-=4mm,y-=2mm}]
		[mesh\_topo.py, file]
		]
		]
		[Text
		[\textcolor{gray}{\textit{beinhaltet diesen Text der Masterarbeit}},
		desc, no edge,
		before drawing tree={x-=4mm,y-=2mm}]
		[Masterarbeit.pdf, file]
		[main.tex, file]
		]
		]
	\end{forest}
}
%\section{Dynamische Erkennung der Netzwerktopologie}
%\subsection{service.py}
%\subsection{ping\textunderscore result\textunderscore  joiner.py}
%\subsection{evaluation.py}

%\section{Ausführung und Migration von Diensten}
%\subsection{background\textunderscore handler.py}
%\subsection{service\textunderscore handler.py}
%\subsection{service\textunderscore transporter.py}
%\newpage
%\section{Evaluation}