\documentclass[ngerman, 11pt]{standalone}
\usepackage[ngerman]{babel} 
\usepackage[utf8]{inputenc}
\usepackage[T1]{fontenc}

\usepackage{booktabs}
\usepackage{tabularx}
\pagenumbering{gobble}

\usepackage{printlen}
\begin{document}
	\noindent
	\begin{tabularx}{433.19327pt}{c|c|c|X}
		\toprule
		Parameter&Typ&Standardwert&Beschreibung\\
		\midrule
		-a&String&./adjacency\_list.json& Speicherort der Adjazenzliste.\\
		-p&int&6001&Port des Service Transporters zur Übertragung des Services.\\
		-s&String&./service.py&Speicherort des Services.\\
		-r&Boolean&False&Angabe, ob der Service direkt gestartet werden soll.\\
		-t&Boolean&False&Angabe, ob der Durchlauf ein Test ist.\\
		-u&String&a& Liste nicht erreichbarer Hosts im Netzwerk.\\
		-v&String&a& Liste der möglichen Serverhosts.\\
		-m&Boolean&True&Angabe, ob der Service überhaupt migriert werden soll.\\
		-c&int&30&Zeit zwischen zwei Migrationszyklen.\\
		-g&float&2.0&Migrationsschwelle in Prozent [\%].\\
		\bottomrule
	\end{tabularx}
\end{document}