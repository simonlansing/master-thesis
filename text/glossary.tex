%!TEX root = main.tex
%==============================================================================
% masculine genitive
\glsaddkey
 {mgenitive}% key
 {}% default value
 {\glsentrymgenitive}% no link cs
 {\Glsentrymgenitive}% no link ucfirst cs
 {\glsmgen}% link cs
 {\Glsmgen}% link ucfirst cs
 {\GLSmgen}% link all caps cs

\glsaddkey
{mgenitivepl}% key
{}% default value
{\glsentrymgenitivepl}% no link cs
{\Glsentrymgenitivepl}% no link ucfirst cs
{\glsmgenpl}% link cs
{\Glsmgenpl}% link ucfirst cs
{\GLSmgenpl}% link all caps cs

\newglossaryentry{IoT}{
    type=\acronymtype,
	name={IoT},
	text={IoT},
	long={Internet der Dinge},
    first={Internet der Dinge (IoT)},
    mgenitive={IoT},
    description={Internet der Dinge (engl. Internet of Things)}
}

\newglossaryentry{WSN}{
    type=\acronymtype,
    name={WSN},
    text={WSN},
    first={kabelloses Sensornetzwerk (engl. Wireless Sensor Netzwork (WSN))},
    description={Kabelloses Sensornetzwerk (engl. Wireless Sensor Netzwork)}
}

\newglossaryentry{RFID}{
	type=\acronymtype,
	name={RFID},
	text={RFID},
	first={RFID},
	description={Identifizierung mit Hilfe elektromagnetischer Wellen (engl. Radio-Frequency Identification)}
}

\newglossaryentry{RTT}{
    type=\acronymtype,
    name={RTT},
    text={RTT},
    first={Round-Trip-Time (RTT)},
    plural={RTTs},
    description={Paketumlaufzeit (engl. Round-Trip-Time)}
}

\newglossaryentry{IPv4}{
    type=\acronymtype,
    name={IPv4},
    text={IPv4},
    description={Internet Protokoll Version 4}
}

\newglossaryentry{IPv6}{
    type=\acronymtype,
    name={IPv6},
    text={IPv6},
    description={Internet Protokoll Version 6}
}

\newglossaryentry{MTU}{
    type=\acronymtype,
    name={MTU},
    text={MTU},
    description={Maximale Übertragungseinheit (engl. Maximum Transmission Unit)}
}

\newglossaryentry{MSS}{
	type=\acronymtype,
	name={MSS},
	text={MSS},
	description={Maximale Segmentgröße (engl. Maximum Segment Size)}
}

\newglossaryentry{TTL}{
    type=\acronymtype,    
    name={TTL},
    text={TTL},
    first={Time to live (TTL)},
    description={Time to live}
}

\newglossaryentry{ETX}
{
    type=\acronymtype,
    name={ETX},
    text={ETX},
    first={erwartete Übertragungsrate (engl. Expected Transmission Count (ETX))},
    description={Erwartete Übertragungsrate (engl. Expected Transmission Count)}
}

\newglossaryentry{ETT}
{
	type=\acronymtype,
	name={ETT},
	text={ETT},
	first={erwartete Übertragungszeit (engl. Expected Transmission Time (ETT))},
	description={Erwartete Übertragungszeit (engl. Expected Transmission Time)}
}

\newglossaryentry{SDN}
{
    type=\acronymtype,
    name={SDN},
    text={SDN},
    first={Software-Defined Networking (SDN)},
    mgenitive={SDNs},
    description={Software-Defined Networking}
}

\newglossaryentry{STP}
{
	type=\acronymtype,
	name={STP},
	text={STP},
	first={Spannbaum-Protokoll (engl. Spanning Tree Protokoll (STP)},
	description={Spannbaum-Protokoll (engl. Spanning Tree Protokoll)}
}

\newglossaryentry{ITU}{
    type=\acronymtype,
    name={ITU},
    text={ITU},
    description={Internationale Fernmeldeunion (engl. International Telecommunication Union)}
}

\newglossaryentry{SOA}{
    type=\acronymtype,
    name={SOA},
    text={SOA},
    first={Serviceorientierte Architektur (SOA)},
    description={Serviceorientierte Architektur}
}

\newglossaryentry{IEEE}{
    type=\acronymtype,
    name={IEEE},
    text={IEEE},
    description={Institute of Electrical and Electronics Engineers}
}

\newglossaryentry{API}{
    type=\acronymtype,
    name={API},
    text={API},
    mgenitive={APIs},
    description={Application Program Interface}
}

\newglossaryentry{CPU}{
	type=\acronymtype,
	name={CPU},
	text={CPU},
	plural={CPUs},
	description={Central Processing Unit}
}

\newglossaryentry{RAM}{
	type=\acronymtype,
	name={RAM},
	text={RAM},
	first={RAM},
	description={Arbeitsspeicher (engl. Random-Access Memory)}
}

\newglossaryentry{JDK}{
	type=\acronymtype,
	name={JDK},
	text={JDK},
	first={Java Development Kit (JDK)},
	description={Java Development Kit}
}

\newglossaryentry{JRE}{
	type=\acronymtype,
	name={JRE},
	text={JRE},
	first={Java-Laufzeitumgebung (engl. Java Runtime Environment (JRE))},
	description={Java-Laufzeitumgebung (engl. Java Runtime Environment)}
}

\newglossaryentry{ComSys}{
    type=\acronymtype,
    name={ComSys},
    text={ComSys},
    first={Communication and Networked Systems (ComSys)},
    description={Communication and Networked Systems}
}

\newglossaryentry{MIOT}{
    type=\acronymtype,
    name={MIOT},
    text={MIOT},
    first={Münster Internet of Things (MIOT)},
    description={Münster Internet of Things}
}

\newglossaryentry{DNS}{
    type=\acronymtype,
    name={DNS},
    text={DNS},
    first={Domain Name System (DNS)},
    description={Domain Name System}
}

\newglossaryentry{NTP}{
    type=\acronymtype,
    name={NTP},
    text={NTP},
    first={Network Time Protocol (NTP)},
    description={Network Time Protocol}
}

\newglossaryentry{TCP}{
    type=\acronymtype,
    name={TCP},
    text={TCP},
    first={TCP},
    description={Transmission Control Protocol}
}

\newglossaryentry{UDP}{
    type=\acronymtype,
    name={UDP},
    text={UDP},
    first={UDP},
    description={User Datagram Protocol}
}

\newglossaryentry{NRO}{
	type=\acronymtype,
	name={NRO},
	text={NRO},
	first={Network Resource Optimization (NRO)},
	description={Network Resource Optimization}
}

\newglossaryentry{VM}{
	type=\acronymtype,
	name={VM},
	text={VM},
	first={Virtuelle Maschine (VM)},
	description={Virtuelle Maschine (engl. Virtual Machine)}
}

\newglossaryentry{VLAN}{
	type=\acronymtype,
	name={VLAN},
	text={VLAN},
	description={Virtual Local Area Network}
}

\newglossaryentry{QoS} {
    type=\acronymtype,
    name={QoS},
    text={QoS},
    first={Quality of Service (QoS)},
    description={Quality of Service}
}

\newglossaryentry{Fog Computing}{
	name={Fog Computing},
	text={Fog Computing},
	first={Fog Computing},
	mgenitive={Fog Computings},
	description={Das Fog Computing stellt eine Brücke zwischen Cloud Computing und dem Internet der Dinge dar, bei der die Rechenkapazität für das Internet der Dinge weit verteilt am \glqq Rande des Netzwerks\grqq{} bereitgestellt wird. Die Eigenschaften des Fog Computings sind geringe Latenzen, Echtzeit, Standortkenntnis, geografisch weite Verteilung, Mobilität und eine große Anzahl Objekte \citep{bonomi2012fog}}
}

\newglossaryentry{Cloud}{
	name={Cloud},
	text={Cloud},
	first={Cloud},
	description={Die Hardware und Software in einem Rechenzentrum wird als die Cloud bezeichnet \citep{armbrust2010view}. Verwendung findet sie bei dem Cloud Computing, bei dem eine schnelle Bereitstellung von Software, Entwicklungswerkzeuge, Hard- und Software-Infrastruktur, Rechenleistung oder Speicherkapazitäten von Anbietern als Dienstleistungen über das Internet erfolgt und auf Anforderungen von Kunden verwendet und wieder freigegeben werden kann. Sowohl die Anwender dieser Ressourcen als auch die Kunden müssen sich nicht um die unterliegende Cloud-Infrastruktur kümmern \citep{mell2011nist}. Für mehr Informationen vgl.~Kapitel~\ref{cha:motivation}}
}

\newglossaryentry{SaaS} {
	name={SaaS},
	text={SaaS},
	first={Software as a Service (SaaS)},
	description={Software as a Service bezeichnet die Bereitstellung von Anwendungssoftware in einer Cloud-Infrastruktur, die über eine Vielzahl von Clients mittels schmalen Client-Interfaces, APIs oder Webbrowsern erreichbar ist \citep{mell2011nist}}
}

\newglossaryentry{PaaS} {
	name={PaaS},
	text={PaaS},
	first={Platform as a Service (PaaS)},
	description={Platform as a Service bezeichnet die Bereitstellung von Kunden entwickelter Anwendungsapplikationen in einer Cloud-Infrastruktur, die mithilfe der vom Anbieter unterstützen Programmiersprachen, Bibliotheken und Entwicklungsumgebungen entwickelt wurden. Die Kunden haben zumeist die Möglichkeit die bereitgestellten Anwendungsapplikationen und Laufzeitumgebungen zu verwalten \citep{mell2011nist}}
}

\newglossaryentry{IaaS} {
	name={IaaS},
	text={IaaS},
	first={Infrastructure as a Service (IaaS)},
	description={Infrastructure as a Service bezeichnet die Bereitstellung von Rechenleistung, Speicherkapazitäten, Netzwerken und anderen grundlegenden Rechenressourcen, in der die Kunden jegliche Software, Betriebssysteme und Anwendungsapplikationen bereitstellen und verwalten können \citep{mell2011nist}}
}

\newglossaryentry{CoAP}{
    name={Constrained Application Protocol (CoAP)},
    text={CoAP},
    first={Constrained Application Protocol (CoAP)},
    description={Das Constrained Application Protocol ist ein Webtransfer-Protokoll für die Verwendung in einer Umgebung aus Objekten und Netzwerken, die in ihren Ressourcen eingeschränkt sind. Während die Objekte oft eine geringe Speicherkapazität aufweisen, sind die Netzwerke mit einer hohen Paketfehlerrate und sehr geringen Durchsatz gekennzeichnet \citep{RFC7252}}
}

\newglossaryentry{Mehrkosten}{
    name={Mehrkosten},
    text={Mehrkosten},
    first={Mehrkosten (engl. Overhead)},
    description={Als Mehrkosten (engl. Overhead) im Sinne der Datenübertragung werden die Daten bezeichnet, die zusätzlich zu den reinen Nutzdaten gesendet werden müssen. Dazu zählen beispielsweise neben den Headern von Protokollen auch verschiedene Kontroll- und Signalisierungsdaten, die zur Realisierung einer stabilen Verbindung beitragen}
}

\newglossaryentry{CP}{
	name={Steuerungsebene},
	text={Steuerungsebene},
	first={Steuerungsebene (engl. Control Plane)},
	description={Die Steuerungsebene (engl. Control Plane) ist ein Begriff aus dem Soft\-ware-Defined Networking, bei der aus Routing-Tabellen des Netzwerkes die For\-war\-ding-Tabellen für die Weiterleitungsebene (engl. Forwarding Plane) erstellt werden \citep{etherealmind2011openflow}. Für mehr Informationen vgl.~Kapitel~\ref{sec:emulationsumgebung_software_defined_network}}
}

\newglossaryentry{FP}{
	name={Weiterleitungsebene},
	text={Weiterleitungsebene},
	first={Weiterleitungsebene (engl. Forwarding Plane)},
	description={Die Weiterleitungsebene (engl. Forwarding Plane respektive Data Plane) ist ein Begriff aus dem Software-Defined Networking. Kommen Ethernet Frames an einem Switch-Interface eines Netzwerkgerät an, kümmert sich die Weiterleitungsebene um die Weiterleitung dieser zu einem Ausgang. Über Forwarding-Tabellen können Regeln für die Weiterleitung definiert werden \citep{etherealmind2011openflow}. Für mehr Informationen vgl.~Kapitel~\ref{sec:emulationsumgebung_software_defined_network}}
}

%\newglossaryentry{Softwareverteilung}{
%	name={Softwareverteilung},
%	text={Softwareverteilung},
%	first={Softwareverteilung (engl. software deployment)},
%	description={}
%}

\newglossaryentry{Fork}{
	name={Fork},
	text={Fork},
	description={Eine Abspaltung (engl. Fork) ist in der Softwareentwicklung ein Entwicklungszweig eines Projektes, bei dem der Quellcode teilweise oder vollständig vom Projekt übernommen und unabhängig weiterentwickelt wird}
}

\newglossaryentry{Spanning Tree Protokoll}{
	name={Spanning Tree Protokoll},
	text={Spanning Tree Protokoll},
	description={Das Spannbaum-Protokoll (engl. Spanning Tree Protokoll) ist ein Netzwerkprotokoll, das eine Schleifenfreiheit in logischen Netzwerktopologien garantiert und durch Schleifen erzeugte Broadcast-Stürme verhindert}
}

\newglossaryentry{Lastverteilung}{
    name={Lastverteilung},
    text={Lastverteilung},
    first={Lastverteilung (engl. load balancing)},
    description={Eine Lastverteilung (engl. load balancing) verbessert die Verteilung von Arbeitsbelastungen über mehrere Ressourcen mit Rechenleistung, wie Computer. Das Ziel ist eine Optimierung der Ressourcen, wie die Maximierung des Durchsatzes, Minimierung von Latenzen und die Vermeidung der Überlastung einzelner Ressourcen. Dafür wird jedoch eine spezielle Hard- oder Software benötigt, die sich mit der Koordinierung beschäftigt}
}

\newglossaryentry{Taktile Internet}{
    name={Taktile Internet},
    text={Taktile Internet},
    first={Taktile Internet (engl. Tactile Internet)},
    mgenitive={Taktilen Internets},
    description={Unter dem Taktilen Internet versteht man eine extrem kurze und für den Menschen nicht wahrnehmbare Reaktionszeit (weniger als eine Millisekunde) einer via Internet gesteuerten Anwendung. Damit werden zukünftige Anwendungen, etwa in der Telemedizin oder Car2Car-Kommunikation, möglich. Vor allem aber wird so das Internet der Dinge realisiert \citep{fraunhofer2016taktiles}}
}

\newglossaryentry{Echtzeit}{
    name={Echtzeit},
    text={Echtzeit},
    first={Echtzeit (engl. real-time)},
    description={Unter Echtzeit (engl. real-time) versteht man die unmittelbare Reaktion ohne spürbare Verzögerung einer Anwendung. Eine genaue zeitliche Festlegung gibt es für den Begriff allerdings nicht \citep{fraunhofer2016taktiles}}
}

\newglossaryentry{Diensteanbieter}{
    name={Diensteanbieter},
    text={Diensteanbieter},
    first={Diensteanbieter (engl. Service Provider)},
    mgenitive={Diensteanbieters},
    description={Ein Dienstanbieter (engl. Service Provider) im Sinne des Cloud Computings stellt Dienstleistungen, wie Server-Infrastrukturen und einhergehende Rechenleistungen und Speicherkapazitäten, bereit. Da ein Anbieter es mehrere Kunden gleichzeitig anbieten kann, können er die Kapazitäten besser skalieren. Infolgedessen sinken die Kosten der Dienstleistungen für die Kunden im Vergleich einer eigenen Infrastruktur}
}

%\newglossaryentry{Broker}{
%	name={Broker},
%	text={Broker},
%	description={}%Broker in SOA beschreiben (Broker (service-oriented architecture))}
%}

\newglossaryentry{Mesh}{
    name={Mesh},
    text={Mesh},
    first={Mesh},
    description={Ein Mesh-Netzwerk ist eine Netzwerktopologie, bei der jeder Netzwerkknoten mit anderen Netzwerkknoten direkt verbunden ist. Alle Knoten kooperieren miteinander und können die Daten innerhalb des Netzwerkes bis zum gewünschten Ziel weiterleiten}
}

\newglossaryentry{Broadcast}{
    name={Broadcast},
    text={Broadcast},
    first={Broadcast},
    description={Ein Broadcast ist eine Methode zur gleichzeitigen Übermittlung einer Nachricht an alle Teilnehmer eines Netzwerkes}
}

\newglossaryentry{Ad-hoc-Netzwerk}{
    name={Ad-hoc-Netzwerk},
    text={Ad-hoc-Netzwerk},
    first={Ad-hoc-Netzwerk},
    plural={Ad-hoc-Netzwerke},
    mgenitivepl={Ad-hoc-Netzwerken},
    description={Ein Ad-hoc-Netzwerk ist ein Zusammenschluss von Netzwerkknoten, die selbstständig ein gemeinsames Mesh-Netzwerk bilden und konfigurieren sowie ohne einen zentralen Zugangspunkt miteinander kommunizieren können \citep{wittenburg2010service}}
}

\newglossaryentry{Netzzugangsschicht}{
    name={Netzzugangsschicht},
    text={Netzzugangsschicht},
    first={Netzzugangsschicht (engl. Link Layer)},
    description={Die Netzzugangsschicht (engl. Link Layer) ist die unterste Schicht der Internetprotokollfamilie. Sie verpackt die zu sendenden Pakete in Frames und garantiert eine möglichst fehlerfreie Übertragung zwischen benachbarten Netzwerkknoten}
}

\newglossaryentry{Internetschicht}{
    name={Internetschicht},
    text={Internetschicht},
    first={Internetschicht (engl. Internet Layer)},
    description={Die Internetschicht (engl. Internet Layer) vermittelt die Pakete durch das Netzwerk, indem sie zunächst durch die Wahl von Routen das nächste Zwischenziel für die Pakete ermittelt und die Pakete anschließend dorthin weiterleitet. Im zentralen Fokus liegt hier das Internet Protokoll in seinen verschiedenen Versionen (IPv4, IPv6)}
}

\newglossaryentry{Hop}{
    name={Hop},
    text={Hop},
    plural={Hops},
    mgenitive={Hops},
    description={Ein Hop ist ein Wegabschnitt zwischen einem Start- und einem Zielknoten für ein im Netzwerk weitergeleitetes Paket, wobei jede Bridge, jeder Router und jedes Gateway als ein Hop zählt}
}

\newglossaryentry{NAT}{
    name={NAT},
    text={NAT},
    description={Die Netzwerkadressübersetzung (engl. Network Address Translation) ist eine Methode, bei der die IP-Adresse eines Datenpaketes während der Übertragung vom Router durch eine andere ersetzt wird}
}

\newglossaryentry{Gateway}
{
    name={Gateway},
    text={Gateway},
    first={Gateway},
    plural={Gateways},
    description={Ein Gateway ist ein Gerät einer Netzwerk-Infrastruktur, das zwei Netzwerke unterschiedlicher Technologien miteinander verbindet und die Kommunikation durch eine gegenseitige Übersetzung des Inhaltes ermöglicht}
}

\newglossaryentry{Virtual Network Emulator}
{
    name={Virtual Network Emulator},
    text={Virtual Network Emulator},
    mgenitive={Virtual Network Emulators},
    description={Ein Virtual Network Emulator ist eine Testumgebung, die den Aufbau eines virtuelles Netzwerkes und so das Testen der Performance realer Applikationen ermöglicht}
}

\newglossaryentry{Durchsatz}
{
    name={Durchsatz},
    text={Durch\-satz},
    first={Durchsatz (engl. Throughput)},
    mgenitive={Durchsatzes},
    description={Der Durchsatz einer Verbindung ist ein Maß für die maximale Menge erfolgreich übertragener Daten pro Zeiteinheit}
}

\newglossaryentry{ICMP}{
    name={ICMP},
    text={ICMP},
    first={ICMP},
    description={Das Internet Control Message Protocol (ICMP) ist ein unterstützendes Protokoll der Internetprotokollfamilie und wird von Netzwerkgeräten zur Übermittlung von Informations- und Fehlermeldungen verwendet. Das Diagnosewerkzeug ping verwendet ICMP zur Übertragung von \glqq Echo request\grqq{} und \glqq Echo reply\grqq{} Paketen}
}

\newglossaryentry{M2M}{
    name={M2M},
    text={M2M},
    first={Machine-to-Machine (M2M)},
    description={Machine-to-Machine (M2M) steht für den automatisierten Informationsaustausch (meist via Internet) ohne menschlichen Eingriff zwischen Maschinen, Automaten und Fahrzeugen untereinander oder mit einer zentralen Leitstelle \citep{fraunhofer2016taktiles}}
}

\newglossaryentry{PID}{
    name={PID},
    text={PID},
    first={Prozess ID (PID)},
    description={Eine Prozess ID ist eine eindeutige Nummer, die ein Betriebssystem zur Identifizierung eines einzelnen Prozesses verwendet}
}

\newglossaryentry{Entwurfsmuster}{
    name={Entwurfsmuster},
    text={Entwurfsmuster},
    first={Entwurfsmuster (engl. Design Pattern)},
    description={Ein Entwurfsmuster (engl. Design Pattern) ist eine wiederverwendbare Lösungsschablone für bekannte Probleme bei dem Entwurf einer Softwarearchitektur \citep{gamma2015designpatterns}}
}

\newglossaryentry{Verhaltensmuster}{
    name={Verhaltensmuster},
    text={Verhaltensmuster},
    first={Verhaltensmuster (engl. behavioral pattern)},
    mgenitive={Verhaltensmusters},
    description={Ein Verhaltensmuster ist eine Kategorie von Entwurfsmustern. Sie charakterisiert die Art und Weise der Interaktion von Klassen und Objekten sowie die Verteilung der Zuständigkeiten \citep{gamma2015designpatterns}}
}

\newglossaryentry{Wertobjekt}{
    name={Wertobjekt},
    text={Wertobjekt},
    first={Wertobjekt (engl. Value Object)},
    description={Ein Wertobjekt (engl. Value Object) ist ein einfaches Objekt, dessen Gleichheit nicht auf seine Identität basiert, sondern auf seinen Wert. Es ist vergleichbar mit den primitiven Datentypen einiger Programmiersprachen \citep{fowler2002patterns}}
}

\newglossaryentry{Timer}{
    name={Zeitgeber (engl. Timer)},
    text={Timer},
    first={Zeitgeber (engl. Timer)},
    mgenitive={Timers},
    description={Ein Zeitgeber (engl. Timer) realisiert zeitbezogene Funktionen durch das Messen von Zeitintervallen}
}